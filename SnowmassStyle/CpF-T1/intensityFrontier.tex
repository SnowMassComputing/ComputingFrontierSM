Circular accelerators are a central feature in nearly all
proposed plans for the future of the Intensity Frontier. Since
the intensity-limiting effects in accelerators are collective in
nature, accurate studies of potential collective effects are
critical portions of the design process. The three primary
collective effects in question are space charge, impedance and
electron cloud. Ongoing studies of these effects in the Fermilab
Booster and Main Injector provide concrete examples of the types
of studies that will be necessary for any Intensity Frontier
circular accelerator.  Useful simulations have three main
requirements. The first requirement is a model of the accelerator
itself that contains enough detail to effectively capture the
physical effects leading to losses. Important details include
realistic apertures, magnet fringe fields, misalignments, etc.
The second requirement is a simulated time period long enough to
capture the various loss mechanisms that come into play. The
third is an overall level of fidelity in the simulation great
enough to have confidence in the final results. These three
requirements together put constraints on both software, which
must accommodate the complexity required for realistic models,
and computing hardware, which must be capable of delivering
detailed simulations in a timely manner.  For simulations of the
Main Injector, the first needs are accurate simulations of space
charge and impedance combined with a detailed model of the
accelerator. Space charge and impedance-related simulation topics
to address in the Main injector include space-charge tune shifts
and tune spreads. These studies will lead into studies of the
variation of operational parameters to minimize losses, which
will require many runs as the parameter space is scanned. Further
studies include simulations of injection beam painting and
mitigation techniques such as electron lenses. In all cases, the
simulation program must be benchmarked against corresponding beam
studies (for example at facilities such as ASTA and UMER). 
Simulations of relatively new technologies
such as electron lenses are especially important to pair with
experimental measurements to validate models.

Electron cloud development and its effect on beam dynamics is
another subject of concern in the Main Injector. A program to
simulate electron cloud development is ongoing; simulations of electron
cloud effects in beam dynamics are planned. Because the electron
cloud phenomenon is the product of a complex set of physical
effects, an experimental program to study these effects and
validate the simulations is also required. Such an effort has
started, but will require more work to reach the accuracy needed.
Simulations of space charge and impedance effects in the Booster
are also of great importance. In the Booster, inter-bunch
communication through impedance has been shown to critically
depend on the number of bunches present in the simulation. These
results indicate that simulations containing the entire 84-bunch
filled Booster ring are necessary. Such simulations have all the requirements
of the single-bunch simulations with additional computational
complexity proportional to the number of bunches. Such
simulations require supercomputer resources.

The computational requirements of these simulation programs can
be estimated for the Main Injector. A single Main Injector
revolution takes roughly 11 microseconds. The time scale
associated with losses is half a second. The simulations must
therefore address ~50,000 turns. Simulations must take into
account the variation of the beta function around the ring,
sampling several times per period. There are 104 such periods in
the main injector, meaning that the simulations must contain on
the order of 500 steps per turn. Detailed simulations must
therefore contain tens of millions of time steps. The Intensity
Frontier puts strict limits on acceptable losses. If we take 1e-4
has the acceptable loss limit, and we require 1\% accuracy in our
simulations of loss, we require 1e8 macroparticles. Such
simulations are appropriate for today's supercomputers; when the
additional factors associated with multi-bunch simulations are
added to the mix, simulations will require the very cutting-edge
of current supercomputer technology.

Although we used IF accelerators as an example to
describe the code capabilities and size of computation necessary to move in the
future, similar requirements exist for EF hadron colliders, such as VLHC and 
HL-LHC, although self-fields are not important, and beam-beam effects could be
important.  In addition, for operation of such IF or EF machines of the future
control room feedback capabilities is desirable (because of the loss 
implications).  The necessary analysis workflow
and synthetic diagnostic tools would be similar to those used by HEP
experiments today, since they will have to model the beam detector
response and maintain and correlate the information of the
simulated physics variables to those smeared by the model of the
diagnostics. 

