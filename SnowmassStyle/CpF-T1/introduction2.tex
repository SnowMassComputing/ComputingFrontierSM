Particle accelerators are critical to scientific discovery both nationally and worldwide. The development and optimization of accelerators are essential for advancing our understanding of the fundamental properties of matter, energy, space and time. Modeling of accelerator components and simulation of beam dynamics are necessary for understanding and optimizing the performance of existing accelerators, for optimizing the design and cost effectiveness of future accelerators, and for discovering and developing new acceleration techniques and technologies.

The requirements for high-fidelity computer simulations of accelerator systems and accelerator components are driven by the need to develop and optimize new accelerator concepts and design machines based on these concepts, and maximize the performance of accelerators based on existing concepts and technologies.  For  Energy Frontier applications this means supporting the development of new techniques that will increase the accelerating gradients so future machines are more compact and less costly. The options considered in our study include acceleration in plasma structures, using either laser or beam driven wakefields, dielectric structures driven by lasers or RF (GHz), the development of new lepton collider designs such as muon colliders and two-beam acceleration, and optimization of existing technologies such as superconducting rf cavities. For  Intensity Frontier, simulations are essential in developing and optimizing integrated designs in order to minimize beam losses due to instabilities caused either by beam self-interactions or by interactions of the beam with the accelerator structures or other media present in the beam pipe.  This  includes both designing mitigation techniques and determining optimal operational parameters.  Hadron colliders at the Energy Frontier have similar requirements, although self-interactions are not important and beam-beam interactions (which are similarly computationally intensive) have to be included.  

Simulations of accelerators for both the Energy and the Intensity frontier are computationally demanding because they often involve a wide range of time and length scales and a wide spectrum of interoperating physics components. For example, high intensity proton drivers of the order of $10^3$ m, operating at an EM wavelength of $10^2$--$10$ m with components of the order of $10$--$1$ m must resolve particle bunches of the order of $10^{-3}$ m. Similarly, laser-plasma accelerators (LPA) of the order of $1$ m in length must resolve laser wavelength and electron bunch size of the order of $1$ $\mu$m.

Most of software for accelerator science are already  highly parallelized and scalable to $> 10$k cores on HPC. They use a wide variety of numerical models, such as electrostatic (multigrid, AMR multigrid, spectral), electromagnetic (finite difference, finite element direct and hybrid, extended stencil finite-difference, AMR finite-difference), quasi-static (spectral), and  matrix solvers, Particle in Cell, meshin and other libraries, and a variety of analysis tools. In addition, there are ongoing R\&D efforts to port these numerical models on new architectures such as GPU based machines.  Although the physics models implemented in today's simulation tools utilize the above sophisticated infrastructure, because of the size of the computation, often "single physics" or "few physics" models are included in a run. The different physics effects are studied separately, as if they were independent.  This is not the case in general, affecting our ability to find optimal design and operational parameters.  More efforts are needed to integrate multiple physics for more accurate simulations, with the ability to utilize massive computing resources beyond the capabilities of today. In the energy frontier, where single components of the accelerator are simulated separately, end-to-end simulations and integration between components is needed.  For example, plasma based accelerators simulations must be advanced from modeling current experiments at the 10 GeV and 0.1 micron emittance level to future collider concepts involving 100s of stages at the 0.01 micron emittance level, which also requires integration of additional physical models such as scattering and radiation. For high-intensity circular proton machines, a large number of macro-particles ($\sim 10^9$) must be used in the simulations in order to accurately represent \% level losses. In addition, detailed models of important components relevant to all frontier applications are missing from our simulation toolkits because of prohibitive computational cost and complexity (for example target modeling, including Gas dynamics, MHD, and heat loading/dissipation must be integrated to our toolkit). 