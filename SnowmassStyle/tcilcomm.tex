% Macros used for TCI styles released with Scientific Word 1.1.
%Copyright (C) 1992-93 TCI Software Research, Inc.
\typeout{TCILCOMM Macros for TCI styles <08 Apr 93>.}
%
% Revisions:
% 08 Apr 93 - GP change line length to under 80 characters
% 30 Mar 92 - AO added \setmoremathsizes and hooks to, e.g. \@xpt,
%             for support of bold Greek
%
%%ltg added this to try to get 12pt to come through
%\DeclareOption{12pt}{\renewcommand\@ptsize{2}}
%%end ltg

\def\@ptoptions{%
 \let\@elt\relax\@for\@tempa:=\@optionlist\do{%
  \expandafter\let\expandafter\@tempb\csname ds@\@tempa\endcsname
  \expandafter\let\expandafter\@tempc\csname ds@11pt\endcsname
  \ifx\@tempb\@tempc\else
   \expandafter\let\expandafter\@tempc\csname ds@12pt\endcsname
   \ifx\@tempb\@tempc\else\let\@tempb\relax\fi
   \fi
  \@tempb
  }%
 }%
%
% read in style file, but only process the pt size options
%
\def\@ptoptinput#1{%
 \let\@save@optionlist\@optionlist
 \let\@save@options\@options
 \let\@options\@ptoptions
 %
 \input #1
 %
 \let\@optionlist\@save@optionlist
 \let\@options\@save@options
 }%
%
% the <21 Feb 92> version of letter.sty requires:
\ifx\undefined\reset@font
 \def\reset@font{}%
 \fi
%
% Change LaTeX
%
% Add command to append tokens
\newtoks\@temptokenb
\long\def\appdef#1#2{%
 \ifx#1\undefined\let#1\@empty\fi
 \@temptokena\expandafter{#1}\@temptokenb{#2}%
 \edef#1{\the\@temptokena\the\@temptokenb}%
 }%
\long\def\prepdef#1#2{%
 \ifx#1\undefined\let#1\@empty\fi
 \@temptokena\expandafter{#1}\@temptokenb{#2}%
 \edef#1{\the\@temptokenb\the\@temptokena}%
 }%
% Discard contents of \box\@cclv at beginning of document
\appdef\document{{\output{\setbox\z@\box\@cclv}\endgraf\null\eject}}%
% Fix mark register expansion
\def\leftmark{\expandafter\@leftmark\botmark{}{}{}\@nil}%
\def\rightmark{\expandafter\@rightmark\firstmark{}{}{}\@nil}%
\def\@leftmark#1#2#3\@nil{#1}%
\def\@rightmark#1#2#3\@nil{#2}%
%
%\show\raggedright
\newskip\raggedglue
\raggedglue\z@ plus20\p@ minus\z@
\def\raggedright{%
 \let\\=\@centercr\@rightskip\raggedglue\rightskip
 \@rightskip\leftskip\z@\parindent\z@
 }%
%
\def\@centercr{%
 \ifhmode\unskip\else\@badcrerr\fi\@@par
 \@ifstar{\penalty\@M\@xcentercr}{\@xcentercr}%
 }%
%
\def\traceoutput{\tracingoutput2\showboxbreadth\maxdimen
 \showboxdepth\maxdimen}%
%
\newcount\@hour\newcount\@minute\chardef\@x10\chardef\@xv60
\def\@time{%
  \@minute\time\@hour\@minute\divide\@hour\@xv
  \ifnum\@hour<\@x 0\fi\the\@hour:%
  \multiply\@hour\@xv\advance\@minute-\@hour
  \ifnum\@minute<\@x 0\fi\the\@minute
  }%
%
% change \@array to allow \par to mean something within its scope
%
\def\@array [#1]#2{%
 \setbox\@arstrutbox\hbox{\vrule height\arraystretch\ht
  \strutbox depth\arraystretch\dp\strutbox width\z@}%
 \@mkpream{#2}%
 \edef\@preamble{\halign\noexpand\@halignto\bgroup\tabskip
  \z@\@arstrut\@preamble\tabskip\z@\cr}%
 \let\@startpbox\@@startpbox\let\@endpbox\@@endpbox
 \if#1t\vtop\else\if#1b\vbox\else\vcenter\fi\fi
 \bgroup
  %\let\par\relax
  \let\@sharp##
  \let\protect\relax
  \lineskip\z@\baselineskip\z@
  \@preamble
 }%
%
% Expands to hex digit corresponding to the value of the argument
%
\def\hexnumber@#1{\ifnum#1<10 \number#1\else
 \ifnum#1=10 A\else\ifnum#1=11 B\else\ifnum#1=12 C\else
 \ifnum#1=13 D\else\ifnum#1=14 E\else\ifnum#1=15 F\fi\fi\fi\fi\fi\fi\fi}%
%
% define command to set more math sizes.
% Styles like Greekbf will hook in here.
%
\def\setmoremathsizes#1#2#3{}%
%
% define size switch addenda
%  Note: these choices agree with those in lfonts.tex, even if incorrect.
%
\prepdef\@vpt{\setmoremathsizes{fiv}{fiv}{fiv}}%
\prepdef\@vipt{\setmoremathsizes{six}{six}{six}}%
\prepdef\@viipt{\setmoremathsizes{sev}{fiv}{fiv}}%
\prepdef\@viiipt{\setmoremathsizes{egt}{six}{fiv}}%
\prepdef\@ixpt{\setmoremathsizes{nin}{six}{fiv}}%
\prepdef\@xpt{\setmoremathsizes{ten}{sev}{fiv}}%
\prepdef\@xipt{\setmoremathsizes{elv}{egt}{six}}%
\prepdef\@xiipt{\setmoremathsizes{twl}{egt}{six}}%
\prepdef\@xivpt{\setmoremathsizes{frtn}{ten}{sev}}%
\prepdef\@xviipt{\setmoremathsizes{svtn}{twl}{ten}}%
\prepdef\@xxpt{\setmoremathsizes{twty}{frtn}{twl}}%
\prepdef\@xxvpt{\setmoremathsizes{twty}{twty}{svtn}}%
%
% The \newfield command defines (like \newenvironment) a user syntax:
%
%   \newfield{foo}
%
% lets you write
%
%   \begin{foo}...\end{foo}
%
% but differs from \newenvironment in that TeX parses the entire content
% of the field and stores it in a macro \the@foo, and then executes \@do@foo.
%
% This means that the content of the field is actually more like an argument
% than the content of an environment. So \verb will not necessarily work
% within a field, and neither will \index. Caveat emptor.
%
% The default procedure for a field is to process its argument within a
% group.  If you want a field to simply store its data away, define its
% procedure to be synonymous with \@empty, as illustrated below.
%
\long\def\@readenvir#1#2\end{%
 \appdef\envirsofar{#2}%
 \@ifnextchar\bgroup{\@checkenvir{#1}}{\@envirenderr}%
 }%
%
\def\@envirenderr{%
 \@latexerr{\string\end\space not followed by environment name}\@eha
 }%
%
\def\@checkenvir#1#2{%
 \def\@tempa{#1}\def\@tempb{#2}%
 \ifx\@tempa\@tempb
  \ifx\envirsofar\@blankspace\let\envirsofar\@empty\fi
  \expandafter\let\csname the@#1\endcsname\envirsofar
  \def\@tempa{\csname @do@#1\endcsname}%
  \else
  \appdef\envirsofar{\end{#2}}%
  \def\@tempa{\@readenvir{#1}}%
  \fi
 \@tempa
 }%
%
\def\newfield#1#2{%
 \expandafter\@ifdefinable\csname#1\endcsname{%
  \expandafter\def\csname#1\endcsname{%
   \endgroup\let\envirsofar\@empty\@readenvir{#1}%
   }%
  \expandafter\let\csname the@#1\endcsname\@empty
  \expandafter\ifx\csname @do@#1\endcsname\relax
   \expandafter\def\csname @do@#1\endcsname{{\csname the@#1\endcsname}}%
   \fi
  }%
 }%
%
\def\@blankspace{ }% SciWord's nil field
%
\def\letparBL{\def\par{\\}}%
\def\letparSP{\def\par{\ }}%
\def\letparQD{\def\par{\quad}}%
\def\letparSL{\def\par{\hbox to1em{\hfil/\hfil}}}%
%
% When \par is processed in the page head, gobble up the rest of the
% argument.
\def\@gobblepar#1\@empty{}%
%
% Physical page size after trim
\newdimen\trimwidth
\newdimen\trimheight
\newdimen\botmargin
\newdimen\marginpartotal
%
\def\adjustpagemargin{% adjust \oddsidemargin by \marginpartotal
 \marginpartotal\marginparwidth\advance\marginpartotal\marginparsep
 \advance\oddsidemargin\marginpartotal
 }%
%
\def\settextheight{% calculate textheight constrained by:
 \textheight\trimheight
 \advance\textheight-\topmargin
 \advance\textheight-\headheight
 \advance\textheight-\headsep
 \advance\textheight-\footskip
 \advance\textheight-\botmargin
 }%
%
\def\vcenterpagemargin{% calculate \topmargin constrained by:
 \topmargin\trimheight
 \advance\topmargin-\textheight
 \advance\topmargin-\headheight
 \advance\topmargin-\headsep
 \advance\topmargin-\footskip
 \divide\topmargin\tw@\botmargin\topmargin
 }%
%
\def\centerpagemargin{% calculate \oddsidemargin constrained by:
 \oddsidemargin\trimwidth
 \advance\oddsidemargin-\textwidth
 \advance\oddsidemargin-\marginparwidth
 \advance\oddsidemargin-\marginparsep
 \divide\oddsidemargin\tw@
 \evensidemargin\oddsidemargin
 \adjustpagemargin
 }%
%
\def\leftpagemargin{% calculate \oddsidemargin constrained by:
 \oddsidemargin\trimwidth
 \advance\oddsidemargin-\textwidth
 \advance\oddsidemargin-\marginparwidth
 \advance\oddsidemargin-\marginparsep
 \advance\oddsidemargin-\evensidemargin
 \adjustpagemargin
 }%
