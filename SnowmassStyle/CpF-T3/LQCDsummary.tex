Facilities for numerical lattice gauge theory are an essential theoretical compliment to the experimental
high-energy physics program.  Lattice-QCD calculations now play an essential role in the search for new physics at the intensity frontier.
They provide accurate results for many of the hadronic matrix elements needed to realize the potential of
present experiments probing the physics of flavor. The methodology has been validated by comparison 
with a broad array of measured quantities, several of which had not been well measured in experiment when the first good lattice calculation became available.  In the next decades, lattice-QCD has the welcome opportunity to play an expanded role in the search for new physics at both the energy and intensity frontiers.  

The USQCD Collaboration, which consists of most theoretical physicists in the U.S. involved in the numerical study of QCD and beyond-the-Standard Model theories using lattice methods, has laid out an ambitious vision for future lattice calculations matched to the experimental priorities of the planned experimental high-energy physics program in the white papers ``Lattice QCD at the Intensity Frontier" and ``Lattice Gauge Theories at the Energy Frontier" \cite{USQCD_IF_whitepaper13,USQCD_EF_whitepaper13}.  These detailed documents present a concrete five-year plan for both the collaboration's foremost scientific goals and the theoretical, algorithmic, and computational strategies for achieving them.

In the U.S., the effort of the lattice gauge-theory community has been supported in an essential way by hardware and software support provided to the USQCD Collaboration.  The USQCD Collaboration's hardware project is up for renewal in 2015, and USQCD is currently in the midst of obtaining CD-0 approval for the LQCD-III project from the DOE.  Achieving the goals outlined in these white papers and meeting the needs of current, upcoming, and future experiments will require continued support of both the national supercomputing centers and of dedicated USQCD hardware through the LQCD-III project, investment in software development through SciDAC funding, and support of postdoctoral researchers and junior faculty through DOE and NSF grants to lab and university lattice gauge theorists.   

The main findings of this report are summarized here:

\begin{itemize}

\item The scientific impact of many future experimental measurements at the energy and intensity frontiers hinge on reliable Standard-Model predictions on the same time
scale as the experiments and with commensurate uncertainties. Many of these predictions require nonperturbative hadronic matrix elements or fundamental QCD parameters that can only be computed numerically with lattice-QCD. The U.S. lattice-QCD community is well-versed in the plans and needs of the experimental high-energy program
over the next decade, and will continue to pursue the necessary supporting theoretical calculations.   Some of the highest priorities are improving calculations of hadronic matrix elements involving quark-flavor-changing transitions which are needed to interpret rare kaon decay experiments, improving calculations of the quark masses $m_c$ and $m_b$ and the strong coupling $\alpha_s$ which contribute significant parametric uncertainties to Higgs branching fractions, calculating the nucleon axial form factor which is needed to improve determinations of neutrino-nucleon cross sections relevant experiments such as LBNE, calculating the light- and strange-quark contents of nucleon which are needed to make model predictions for the $\mu \to e$ conversion rate at the Mu2e experiment (as well as to interpret dark-matter detection experiments in which
the dark-matter particle scatters off a nucleus), and calculating the hadronic light-by-light contribution to muon $g-2$ which is needed to solidify and improve the Standard-Model prediction and interpret the upcoming measurement as a search for new physics.  Lattice field-theory calculations will also increasingly contribute to collider experiments at the LHC 14-TeV run by providing quantitative nonperturbative input for Higgs and other new-physics model building.

\item The successful accomplishment of USQCD's scientific goals requires access to both capacity and capability machines, and hence support for both leadership class facilities and dedicated computing clusters.  Use of leadership-class facilities alone would provide insufficient computational resources needed to complete the planned calculations, and would be unsuitable for the mix of lattice-field-theory job requirements.   USQCD's experience and proven track-record with purchasing, deploying, utilizing, and maintaining dedicated clusters will enable the collaboration to take advantage of future improvements in commodity clusters, such as increased core counts per processor and improved memory and networking bandwidth.   USQCD's five-year computing strategy uses current vendor roadmaps to anticipate the probable evolution of high-performance computing hardware over this time period.  The purchase of new dedicated lattice hardware on an annual basis, however, provides essential flexibility to accommodate changes and developments, and thereby to purchase the most cost-effective machines for lattice-field-theory calculations.

\item The successful utilization of future computing resources requires software that runs efficiently on new computing architectures, and hence support for postdocs and scientific staff to develop lattice-gauge-theory code.  Such positions cannot be supported by grants to lab and university theory groups alone.   The USQCD Collaboration's libraries for lattice-field-theory calculations are publicly available and are used by most of the U.S. community.  USQCD's experience and proven track-record in developing software for diverse machines such as IBM and Cray supercomputers, PC commodity clusters, and GPU-accelerated clusters, will enable the collaboration to fully exploit the computing capacity of future architectures.  

\item Support of USQCD through hardware and software grants, access to leadership-class computing facilities, and funding lab and university theorists, is essential to fully capitalize on the enormous investments in the DOE's high-energy physics and nuclear-physics experimental programs.  Given continued support of the lattice-gauge-theory effort in the U.S. and worldwide, lattice calculations can play a key role in definitively establishing the presence of physics beyond the Standard Model and in determining its underlying structure.

\end{itemize}
