%%  This is the driver file for working group reports contributed 
%%   to the Snowmass 2013 proceedings

%%  This file includes brings in all the necessary files to provide the
%%  format of the Proceedings
%%
%%  D. Hitlin   9/23/03   derived from the BABAR Physics Book format

%%  Please do not change anything in this file, except to include the
%%  name of your file on the next to last line of this file

%%  To use LATEX with this format, you must have the follwing files 
%%  in the same directory as your text source and figure files
%%  tcibook.cls
%%  fancyhea.sty
%%  work.sty
%%  epsfig.sty
%%  workshopsym.tex       This file provides macros for many common symbols
%%                         Using these macros will provide uniformity of notation
%%                         for the basic particle symbols, units, etc.
%%
%%  These provide the page size, type style, headings, etc.


\documentclass{tcibook}
\usepackage{fancyhea}
\usepackage{work}
\usepackage{bm}       %    enables bold math symbols  e.g.  \bm{\gamma}
\usepackage{graphicx}
\usepackage{hyperref}      % hypertext links %%ARXIV
\usepackage{color,xspace,cite}


%%%% \newcommand{\trademark}{\textsuperscript{\texttrademark}\xspace}

\newcommand{\todo}[1]{\textbf{\color{red}#1}}
\newcommand{\registered}{\textsuperscript{\textregistered}\xspace}


\input workshopsymbols.tex      %   standard macros for common HEP terms

\setlength{\headheight}{14pt}

% subsubsections are numbered as well as chapters, sections and subsections.
\setcounter{secnumdepth}{3}

\begin{document}

\def\bibname{References}
\bibliographystyle{plain}

\raggedbottom

\pagenumbering{roman}

\parindent=0pt
\parskip=8pt
\setlength{\evensidemargin}{0pt}
\setlength{\oddsidemargin}{0pt}
\setlength{\marginparsep}{0.0in}
\setlength{\marginparwidth}{0.0in}
\marginparpush=0pt

% The content begins here

\pagenumbering{arabic}


\renewcommand{\chapname}{chap:intro_}
\renewcommand{\chapterdir}{.}
\renewcommand{\arraystretch}{1.25}
\addtolength{\arraycolsep}{-3pt}


\newcommand{\trademark}{\textsuperscript{\texttrademark}\xspace}


%%%%%%%%%%%%%%%%%%%%%%%%%%%%%%%%%%%%%%%%%%%%%%%%%%%
%%%%%%%%%%%%%%%%%%%%%%%%%%%%%%%%%%%%%%%%%%%%%%%%%%%
%%%     All of your files should be in a subdirectory.  Here the
%%%     subdirectory is called Magnetism  .   The title of your
%%%     report should be   wgreport.tex in that subdirectory.  Input
%%%     that file here
%%%%%%%%%%%%%%%%%%%%%%%%%%%%%%%%%%%%%%%%%%%%%%%%%%%%
%%%%%%%%%%%%%%%%%%%%%%%%%%%%%%%%%%%%%%%%%%%%%%%%%%%

\input CpF-T4/wgreport.tex 
%\input ChargedLeptons/wgreport.tex
%\input HeavyPhotons/wgreport.tex

%%%%%%%%%%%%%%%%%%%%%%%%%%%%%%%%%%%%%%%%%%%%%%%%%%
%%%%%%%%%%%%%%%%%%%%%%%%%%%%%%%%%%%%%%%%%%%%%%%%%%
%%%   Your subdirectory (here Magnetism) should include
%%%    the files:
%%%           wgreport.tex
%%%           authorlist.tex
%%%         and all needed figures in pdf format
%%%%%%%%%%%%%%%%%%%%%%%%%%%%%%%%%%%%%%%%%%%%%%%%%%%%
%%%%%%%%%%%%%%%%%%%%%%%%%%%%%%%%%%%%%%%%%%%%%%%%%%%%


\end{document}



