\section{Software Development, Staffing and Training}

Success of HEP science will continue to depend critically on computing.
Managing the human activities (software development and management,
training and staffing) is an important part of that.  
Based upon our own experiences, and from
discussions with members of the HEP community,
we have identified the following main goals for the next decade in the area
of software, staffing and training:

\begin{itemize}
\item Goal: To maximize the scientific productivity of our community
in an era of reduced resources, we must use
software development strategies and staffing models that will result in products
that are generally useful for the wider HEP community.
\item Goal: We must respond to the evolving technology market, especially
with respect to computer processors, by
developing and evolving software that will perform with optimal efficiency
in future computing systems.
\item Goal: We must insure that our developers and users will have the
training needed to create, maintain, and use the increasingly complex software
environments and computing systems that will be part of future HEP projects.
\end{itemize}

Some specific recommendations we feel will help achieve these goals are:

\begin{itemize}
    \item Software Management, Toolkits and Reuse
    \begin{itemize}
        \item Continue to support established toolkits (Geant4, ROOT, ...)
        \item Encourage the creation of new toolkits from existing successful common software (generators, tracking, ...)
        \item  Allow flexible funding of software experts to facilitate transfer of software and sharing of technical expertise between projects
        \item Facilitate code sharing through open-source licensing and use of publicly-readable repositories.
        \item Consolidate and standardize software management tools to minimize cross-experimental "friction"
    \end{itemize}

    \item Software Development for new Hardware Architectures
    \begin{itemize}
        \item Significant investments in software are needed to adapt to the evolution of computing processors, both as basic R\&D into appropriate
techniques and as re-engineering "upgrades"
        \item New software should be designed, and existing software rengineered, to expose parallelism at multiple levels
        \item Develop flexible software architectures that can exploit efficiently a variety of possible future hardware options
    \end{itemize}    

    \item Staffing
      \begin{itemize}
          \item Recognize software efforts as sub-projects of the experiment
          \item Integrate computing professionals as part of the experimental team, over the life of the experiment
          \item Integrate software professionals with scientist developers to insure software meets both the technical and physics performance needs of the experiment
      \end{itemize}

    \item Training
    \begin{itemize}
        \item Use certification to document expertise and encourage learning new skills
        \item Encourage training in software and computing as a continuing physics activity
        \item Use mentors to spread scientific software development standards
        \item Involve computing professionals in the training of scientific domain experts
        \item Use online media to share training
        \item Use workbooks and wikis as evolving, interactive software documentation
        \item Provide young scientists with opportunities to learn computing and software skills that are marketable for non-academic jobs
    \end{itemize}

\end{itemize}
