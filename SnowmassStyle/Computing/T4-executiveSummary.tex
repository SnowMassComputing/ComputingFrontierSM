
\section{Computing for Perturbative QCD}
\label{chap:PQCD}


This workshop provided a framework for implementing higher order  
calculations in a standardized computing environment made available 
by DOE at the National Energy Research Scientific Computing Center
(NERSC)~\cite{NERSC}.  Resource requirements were determined for the
calculation of important background and signal reactions at the
LHC, including higher order QCD and electroweak (EW) effects. Prototypical results 
are listed in Table~\ref{tab:summary} and have been summarized in a 
white paper~\cite{HPCWP}.

Different High Performance Computing (HPC) environments were tested
during this workshop and their suitability for perturbative QCD calculations 
was assessed. We find that it would be beneficial to make the national HPC 
facilities ALCF~\cite{ALCF}, OLCF~\cite{OLCF} and NERSC~\cite{NERSC} 
generally accessible to particle theorists and
experimentalists in order to enable the use of existing
calculational tools for experimental studies involving extensive
multiple runs without depending on the computer power and manpower
available to the code authors. Access to these facilities will also
allow prototyping the next generation of parallel computer programs
for QCD phenomenology and precision calculations.

The computation of next-to-leading order (NLO) corrections in 
perturbative QCD has been entirely
automated. Resource requirements for NLO calculations determined during 
this workshop can thus be seen as a baseline that enables phenomenology 
during the LHC era. Next-to-next-to-leading order (NNLO) calculations are still performed 
on a case-by-case basis, and their computing needs can only be 
projected with a large uncertainty. It seems clear, however, that cutting edge 
calculations will require access to leadership class computing facilities.

The use of HPC in perturbative QCD applications is currently in
an exploratory phase. We expect that the demand for access to HPC
facilities will continue to grow as more researchers realize the 
potential of parallel computing in accelerating scientific progress. 
At the same time, we expect growing demand for educating young researchers 
in cutting edge computing technology. It would be highly beneficial 
to provide a series of topical schools and workshops related 
to HPC in HEP. They may be co-organized with experiments to foster 
the creation of a knowledge base.

Large-scale distributed computing in Grid environments 
may become relevant for perturbative QCD applications 
in the near future. This development will be accelerated if Computing Grids
can also provide access to HPC facilities and clusters where parallel 
computing is possible on a smaller scale. The Open Science Grid (OSG)~\cite{OSG} 
has taken first steps in this direction, and we have successfully used their
existing interface. The amount of training for new users could be minimized
if the OSG were to act as a front-end to the national HPC facilities
as well as conventional computing facilities.

\begin{table}
  \begin{tabular}{ccc}
    \hline
    Type of calculation & CPU hours per project & projects per year \\
    \hline\hline
    NLO parton level & 50,000 - 600,000 & 10-12\\
    NNLO parton level & 50,000 - 1,000,000 & 5-6\\
    Event generation & 50,000 - 250,000 & 5-8\\
    Matrix Element Method & $\sim$ 200,000 & 3-5\\
    Exclusive jet cross sections & $\sim$ 300,000 & 1-2\\
    Parton Distributions & $\sim$ 50,000 & 5-6\\
    \hline
  \end{tabular}
  \caption{Summary of computing requirements for typical projects
    carried out by the US community~\cite{HPCWP}.
    \label{tab:summary}}
\end{table}

\clearpage

\begin{table}
  \centering
  \begin{tabular}{llrr}
    \hline
    Process & Ref. & \multicolumn{2}{c}{Requirements}\\
    & & CPU [core~h] & Storage [GB] \\
    \hline\hline
    $pp\to W^\pm+5 jets$ & \cite{Bern:2013gka} & 600,000 & 1,500 \\
    $pp\to W^\pm+4 jets$ & \cite{Berger:2010zx} & 100,000 & 200\\
    $pp\to Z+4 jets$ & \cite{Ita:2011wn} & 200,000 & 200\\
    $pp\to Z+3 jets$ & \cite{Berger:2010vm} & 50,000 & 100 \\
    $pp\to 4 jets$ & \cite{Bern:2011ep} & 200,000 & 150\\
    \hline
  \end{tabular}
  \caption{CPU and Storage requirements for calculations on the
    list of important processes identified during the LesHouches
    series of workshops~\cite{AlcarazMaestre:2012vp}.
    Numbers assume cross-checks with at least two independent runs
    and are reported for AMD Opteron\trademark processors running at 2.1~GHz.
    Storage requirements are reported for Root NTuple files which can be used
    to replicate the entire event analysis.
  \label{tab:nlo_wishlist}}
\end{table}

\begin{table}
  \centering
  \begin{tabular}{llrr}
    \hline
    Process & Ref. & Requirements & CPU clock\\
    & & CPU [core~h] & [GHz]\\
    \hline\hline
    $pp\to W/Z$ & \cite{Melnikov:2006di,Li:2012wna} & 50,000 & 2.67 \\
    $pp\to H$ & \cite{Anastasiou:2005qj} & 50,000 & 2.67 \\
    $pp\to t\bar{t}$ & \cite{Baernreuther:2012ws,Czakon:2013goa} & 1,000,000 & 2.27\\
    $pp\to $ jets ($g$ only) & \cite{Ridder:2013mf} & 85,000 & 2.20 \\
    $pp\to H+$jet ($g$ only) & \cite{Boughezal:2013uia} & 500,000 & 2.67 \\
    \hline
  \end{tabular}
  \caption{Summary of computing requirements for NNLO calculations.
    Numbers were obtained on Intel\registered Xeon\registered CPU's.
    \label{tab:nnlo_requirements}}
\end{table}

\begin{table}
  \centering
  \begin{tabular}{llllr}
    \hline
    Process & \multicolumn{2}{c}{$N_{jet}$} & Ref. & CPU [core~h] \\
    & NLO & LO & & \\
    \hline\hline
    $pp\to W^\pm+jets$ & $\le$2 & $\le$4 & \cite{Hoeche:2012yf} & 100,000 \\
    $pp\to h+jets$ & $\le$2 & $\le$3 & \cite{Hoeche:2013xxx} & 150,000 \\
    $pp\to t\bar{t}+jets$ & $\le$1 & $\le$2 & \cite{Hoeche:2013mua} & 250,000 \\
    $pp\to l\bar{\nu}\bar{l}'\nu'$ & $\le$1 & $\le$2 & \cite{Cascioli:2013xxx} & 50,000 \\
    \hline
  \end{tabular}
  \caption{Computing requirements for merged NLO simulations in
    various benchmark processes, using the Sherpa event generator
    and assuming cross-checks with at least two independent runs.
    \label{tab:nlo_merging}}
\end{table}

