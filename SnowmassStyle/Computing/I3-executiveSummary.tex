\section{Computing Frontier Networking}


Research in High Energy Physics (HEP) depends on the availability of reliable, high-bandwidth, feature-rich computer networks for interconnecting instruments and computing centers globally. Most HEP-related data is transported by National Research and Education Networks (NRENs), supplemented by infrastructures dedicated to specific projects. NRENs differ from commercial network providers, because they are optimized for transporting massive data flows generated by large-scale scientific collaborations. In addition, NRENs offer advanced capabilities - such as multi-domain dedicated circuits - which commercial providers do not have an incentive to deploy.

For decades, network traffic generated by HEP has been a primary driver of NREN growth, and HEP requirements have motivated NREN architectures and research activities. In the next ten years and beyond, the productivity of HEP collaborations will continue to depend on an ecosystem of innovative global NRENs. 

HEP collaborations are now accustomed to viewing network transport as a reliable and predictable resource – so much so that data models for ATLAS and CMS have evolved rapidly in response to NREN capabilities – but this state of affairs is not inevitable. Other data-intensive communities have begun to generate large traffic flows and, following the example of HEP, to incorporate high-performance networks into science workflows. As a result of this broad trend toward data intensity across many disciplines, NRENs around the world will be challenged to meet the requirements of large-scale research, and must be adequately resourced in order to continue serving the critical role they have played in the past.  

In support of HEP’s objectives through 2020, basic and applied networking research is necessary in a range of subjects. Critical questions include: 

\begin{itemize}
\item[-] What future architectures will maximize utilization and minimize cost in core and campus networks?
\item[-] How can emerging paradigms such as Software Defined Networking or Named Data Networking be harnessed most effectively to improve HEP science outcomes?
\item[-] Can networks evolve into adaptive, self-organizing, programmable systems that quickly respond to requests of HEP science applications? 
\item[-] If well-tuned host systems (or ensembles of them) have the ability to saturate a single backbone channel, what techniques and architectures can NRENs adopt to maximize data mobility?
\item[-] How will the emerging “complexity challenge” arising from closer integration between networks and applications be managed, especially in the multi-domain, multi-national context?
\item[-] How can diverse networks cooperate – automatically and securely – to offer science-optimized capabilities on a worldwide basis? 
\item[-] Can discovery or automation techniques reduce the need for fragile, manual configuration? 
\item[-] How will networks respond to the operational challenge of deploying  and managing dozens of wavelengths across large geographies under relatively flat funding prospects? 
\item[-] Will post-TCP protocols become useful outside of highly-controlled, “walled garden” demonstrations? 
\item[-] Would computer modeling of applications, networks, and data flows be useful in answering any of these questions?  
\item[-] Will power consumption become a limiting economic or operational factor in this time period? 
\end{itemize}

Recent investments in network research have been too small, and continued underfunding will compromise the ability of HEP collaborations to maximize scientific productivity. Increased research funding, while necessary, is not sufficient; there also needs to be increased attention to the process of translating the results of network research into real-world architectures which NRENs can deploy and manage. Incentives and funding for such ‘translational’ activities are urgently needed. Because network research has now begun to intersect with research in services and applications, cross-disciplinary funding opportunities should also be available.  

A number of cultural and operational practices need to be overcome in order for NRENs (and global cyber infrastructures more generally) to fully succeed: expectations for network performance must be raised significantly, so that collaborations do not continue to design workflows around a historical impression of what is possible; the gap between peak and average transfer rates must be closed; and campuses must deploy secure science data enclaves – or  Science DMZs\cite{DMZ} – engineered for the needs of HEP and other data-intensive disciplines.  Fortunately, each of these trends is currently underway, but momentum must be accelerated.   

Ten years from now, the key applications on which HEP depends will not be fully successful, efficient, or cost-effective if they are run on the Internet as it exists today. During the next decade, research networks need to evolve into programmable instruments – flexible resources which can customized for particular needs, but which exist within a common, integrated, ubiquitous framework that is reliable, robust and trusted for its privacy and integrity. These are major challenges, but they are tractable if funding agencies invest in innovative research, and maintain support for the exponential growth of NREN traffic. 

